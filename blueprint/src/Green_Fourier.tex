\chapter{Fourier transform of Green's function}

\section{Fourier transform of the regularized Green's function}

\begin{lemma}
  \label{lem:regularized_Green_square_summable}
  \uses{def:Green_function}
  \lean{regularizedG.l2}
  \leanok % for the 1-dimensional case only
  For $0 \le r < 1$,
  the regularized Green's function $\Greg{r} \colon \bZ^d \to \bR$ is
  an element of the Hilbert space $\ell^2(\bZ^d, \bC)$ of complex-valued
  square-summable functions on $\bZ^d$.
\end{lemma}
\begin{proof}
  \uses{lem:sum_Green_function}
  Since $\ell^1(\bZ^d, \bC) \subset \ell^2(\bZ^d, \bC)$, it suffices to show
  that $\Greg{r}$ is absolutely summable.
  This follows from Lemma~\ref{lem:sum_Green_function}.
\end{proof}

\begin{definition}
  \label{def:Green_Fourier_transf}
  \uses{def:Green_function, lem:regularized_Green_square_summable}
  \lean{regularizedG₁.hat}
  \leanok % for the 1-dimensional case only
  Let $0 \le r < 1$.
  The \textbf{Fourier transform} of the regularized Green's function
  $\Greg{r} \, \colon \, \bZ^d \to \R$
  is the function
  \begin{align*}
  \Freg{r} \, \colon \, \bR^d \to \C
  \end{align*}
  given by
  \begin{align*}
  \Freg{r} (\theta) = \sum_{x \in \bZ^d} e^{\ii x \cdot \theta} \, \Greg{r}(x) .
  \end{align*}
\end{definition}

\section{Explicit formula for the Fourier transform}

\begin{lemma}
  \label{lem:Markovian_Green_Fourier}
  \uses{def:Green_Fourier_transf, def:iid_random_walk}
  For a time-homogeneous random walk $\RW = \big(\RW(t)\big)_{t \in \bN}$ on $\bZ^d$
  with step distribution $p \colon \bZ^d \to [0,1]$,
  the Fourier transform of the Green's function is
  \begin{align*}
  \Freg{r} (\theta) = \Big(1 - r \sum_{u\in \bZ^d} p(u) e^{\ii u \cdot \theta}\Big)^{-1} .
  \end{align*}
\end{lemma}
\begin{proof}
\ldots
\end{proof}

\begin{lemma}
  \label{lem:SRW_Green_Fourier}
  \uses{def:Green_Fourier_transf, def:simple_random_walk}
  For the simple random walk $\RW = \big(\RW(t)\big)_{t \in \bN}$ on $\bZ^d$,
  the Fourier transform of the Green's function is
  \begin{align*}
  \Freg{r} (\theta) = \frac{1}{1-\frac{r}{d} \sum_{j=1}^d \cos(\theta_j)} .
  \end{align*}
\end{lemma}
\begin{proof}
\uses{lem:Markovian_Green_Fourier}
\ldots
\end{proof}

\section{Inversion of the discrete Fourier transform}

\begin{lemma}
  \label{lem:Green_Fourier_inverse}
  \uses{def:Green_Fourier_transf}
  \lean{regularizedG_eq_integral_regularizedG₁hat}
  \leanok % for the 1-dimensional case only
  For any $x \in \bZ^d$ and $0 \le r < 1$, we have
  \begin{align*}
  \Greg{r} (x)
  = \frac{1}{(2\pi)^d} \iint_{\Fbox} e^{-\ii x \cdot \theta} \, \Freg{r} (\theta) \; \ud^d \theta .
  \end{align*}
\end{lemma}
\begin{proof}
\uses{lem:sum_Green_function}
\ldots
\end{proof}

Recall that we are interested in $\EX[L]$, where $L$ is the number of visits to the
origin by the random walk. Lemma~\ref{lem:Green_function_nonregularized_limit}
states that $\EX[L]$ is the increasing limit of $\Greg{r}(\vec{0})$ as $r \nearrow 1$,
and Lemma~\ref{lem:Green_Fourier_inverse} gives a formula for $\Greg{r}(\vec{0})$
as the integral of the Fourier transform: $\Greg{r}(\vec{0}) = \frac{1}{(2\pi)^d} I_r$,
where
\begin{align}
I_r = \iint_{\Fbox} \Freg{r} (\theta) \; \ud^d \theta .
\end{align}

\begin{corollary}
  \label{cor:recurrence_iff_finite_limit_integral}
  \uses{def:expectation_recurrence, def:Green_Fourier_transf}
  A random walk $\RW = \big(\RW(t)\big)_{t \in \bN}$ on $\bZ^d$
  is expectation recurrent if and only if $\lim_{r \nearrow 1} \, I_r = +\infty$.
  In other words, $\RW$ is expectation transient if and only if
  $\lim_{r \nearrow 1} \, I_r \, < \, +\infty$.
\end{corollary}
\begin{proof}
\uses{lem:Green_Fourier_inverse, lem:Green_function_nonregularized_limit}
\ldots
\end{proof}



% This allows to express the
% \begin{corollary}
%   \label{lem:Green_Fourier_inverse}
%   \uses{def:Green_Fourier_transf}
%   For any $x \in \bZ^d$ and $0 \le r < 1$, we have
%   \begin{align*}
%   \Greg{x} (\theta)
%   = \frac{1}{(2\pi)^d} \iint_{\Fbox} e^{-\ii x \cdot \theta} \, \Freg{r} (\theta) \, \ud^d \theta .
%   \end{align*}
% \end{corollary}
% \begin{proof}
% \ldots
% \end{proof}
