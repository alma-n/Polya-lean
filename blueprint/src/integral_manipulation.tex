\chapter{Treatment of the integral in the Fourier inversion}

In this part, we analyze the integral
\begin{align}
I_r = \iint_{\Fbox} \re \big( \Freg{r} (\theta) \big) \; \ud^d \theta ,
\end{align}
where $\Freg{r} \colon \bR^d \to \bC$ is the Fourier transform of the
regularized Green's function of a random walk $\RW$ on $\bZ^d$. By
Corollary~\ref{cor:recurrence_iff_finite_limit_integral}, the finiteness
of this integral in the limit $r \nearrow 1$ characterizes expectation
transience of $\RW$.

\section{Decomposition of the integral}

The main integral $I_r$ can be decomposed into two parts:
an easy "high frequency part", which contains the contributions of
$\theta$ away from $0$, and a more interesting "low frequency part",
which contains the contributions of $\theta$ near $0$.

\begin{definition}[]
  \label{def:frequency_decomposition}
  \uses{def:Green_Fourier_transf}
  \lean{lowFreqIntegral, highFreqIntegral}
  % \leanok
  For any $\delta > 0$, define the integrals
  \begin{align}
  J_r^{(\delta)} = \; & \iint_{\Fbox \setminus B_\delta} \re \big( \Freg{r} (\theta) \big) \; \ud^d \theta \\
  K_r^{(\delta)} = \; & \iint_{B_\delta} \re \big( \Freg{r} (\theta) \big) \; \ud^d \theta ,
  \end{align}
  where $B_\delta := \set{ \theta \in \bR^d \, \big| \, \|\theta\| < \delta}$ is the
  ball of radius $\delta$ centered at $\vec{0} \in \bR^d$.
\end{definition}

\begin{lemma}
  \label{lem:integral_decomposition}
  \uses{def:decomposition} % , lem:Green_Fourier_inverse_real
  \lean{mainIntegral_eq_add}
  For any $0 < \delta \le \pi$ we can write
  \begin{align*}
  I_r = J_r^{(\delta)} + K_r^{(\delta)} .
  \end{align*}
  % where the two parts are
  % \begin{align}
  % J_r^{(\delta)} = \; & \iint_{\Fbox \setminus B_\delta} \re \big( \Freg{r} (\theta) \big) \; \ud^d \theta \\
  % K_r^{(\delta)} = \; & \iint_{B_\delta} \re \big( \Freg{r} (\theta) \big) \; \ud^d \theta ,
  % \end{align}
  % where $B_\delta := \set{ \theta \in \bR^d \, \big| \, \|\theta\| < \delta}$ is the
  % ball of radius $\delta$ centered at $\vec{0} \in \bR^d$.
\end{lemma}
\begin{proof}
Obvious, since $\Fbox = (\Fbox \setminus B_\delta) \cup B_\delta$
is a disjoint union.
\end{proof}


\section{Dominated convergence away from the origin}

TODO: Define non-degenerate step distribution (essentially $\sum_{u\in \bZ^d} p(u) e^{\ii u \cdot \theta} \ne 1$ for $\theta \ne 0$ modulo periodicity).

\begin{lemma}
  \label{lem:high_frequency_integral}
  \uses{def:frequency_decomposition}
  If $\RW$ is a time-homogeneous Markovian random walk with suitable
  non-degeneracy conditions on its step distribution (to be written down more precisely),
  then for any $0 < \delta \le \pi$ the limit
  \begin{align*}
  \lim_{r \nearrow 1} J_r^{(\delta)}
  \end{align*}
  exists and is finite (limit in $\bR$).
\end{lemma}
\begin{proof}
Under the nondegeneracy conditions, on the compact set $\Fbox \setminus B_\delta$,
the continuous integrand $\theta \mapsto \Freg{r}(\theta)$
is bounded (and therefore dominated by a constant function) and
it has the pointwise limit $\lim_{r \nearrow 1} \Freg{r}(\theta) = \Freg{1}(\theta)$.
It therefore follows from the dominated convergence theorem
that $\lim_{r \nearrow 1} J_r^{(\delta)} = J_1^{(\delta)} \in \bR$.
\end{proof}

\section{Monotone convergence near the origin}

TODO: Think about the best conditions for step distribution under which monotone convergence can be applied (real-valuedness requires symmetricity of the step-distribution?!?).

\begin{lemma}
  \label{lem:low_frequency_integral}
  \uses{def:frequency_decomposition}
  If $\RW$ is a time-homogeneous Markovian random walk with suitable
  symmetricity and integrability conditions on its step distribution (to be written down more precisely),
  then there exists a $\delta_0 > 0$ such that for any $0 < \delta \le \delta_0$, the limit
  \begin{align*}
  \lim_{r \nearrow 1} K_r^{(\delta)}
  \end{align*}
  is increasing and exists in $[0,+\infty]$.
\end{lemma}
\begin{proof}
\ldots
\end{proof}

We can now rephrase the recurrence criterion in terms of only the low frequency integral.

\begin{lemma}
  \label{lem:recurrence_iff_finite_low_frequency}
  \uses{def:expectation_recurrence, def:Green_Fourier_transf, def:frequency_decomposition}
  A nice random walk $\RW = \big(\RW(t)\big)_{t \in \bN}$ on $\bZ^d$
  is expectation recurrent if and only if for any small $\delta>0$
  $\lim_{r \nearrow 1} \, K_r^{(\delta)} = +\infty$.
  In other words, $\RW$ is expectation transient if and only if
  for some small $\delta>0$ we have $\lim_{r \nearrow 1} \, K_r^{(\delta)} \, < \, +\infty$.
\end{lemma}
\begin{proof}
\uses{cor:recurrence_iff_finite_limit_integral, lem:high_frequency_integral}
\ldots
\end{proof}


\section{Characterizing finiteness of the integral for simple random walk}

\begin{lemma}
  \label{lem:SRW_low_frequency_approximation}
  \uses{def:frequency_decomposition}
  For the simple random walk $\RW = \big(\RW(t)\big)_{t \in \bN}$ on $\bZ^d$,
  there exists a small $\delta_0 > 0$
  (something like $\delta_0 = \frac{\pi}{8}$) such that
  for all $\theta \in B_{\delta_0} \setminus \set{0}$
  we have $\re \big( \Freg{r}(\theta) \big) \uparrow \re \big( \Freg{1}(\theta) \big)$
  as $r \uparrow 1$ and
  \begin{align*}
  \re \big( \Freg{1}(\theta) \big) \asymp \frac{1}{\| \theta \|^{2}} .
  \end{align*}
\end{lemma}
\begin{proof}
\uses{lem:SRW_Green_Fourier, lem:low_frequency_integral}
Positivity estimates and Taylor series approximation of the Fourier
transform of the Green's function given explicitly in
Lemma~\ref{lem:SRW_Green_Fourier}. \ldots
\end{proof}

\begin{lemma}
  \label{lem:SRW_low_frequency_convergence}
  \uses{def:frequency_decomposition}
  For the simple random walk $\RW = \big(\RW(t)\big)_{t \in \bN}$ on $\bZ^d$
  with $d > 2$,
  we have $\lim_{r \nearrow 1} \, K_r^{(\delta)} \, < \, +\infty$
  for $\delta > 0$ small enough.
\end{lemma}
\begin{proof}
\uses{lem:SRW_low_frequency_approximation}
By the increasing limit part of
Lemma~\ref{lem:SRW_low_frequency_approximation} and
monotone convergence theorem, we have
\begin{align*}
K_r^{(\delta)} \uparrow K_1^{(\delta)}
  = \iint_{B_\delta} \re \big( \Freg{1} (\theta) \big) \; \ud^d \theta .
\end{align*}
By the second part of Lemma~\ref{lem:SRW_low_frequency_approximation},
$\re \big( \Freg{1}(\theta) \big) \ge C_1 \| \theta \|^{-2}$ (for some positive constant $C_1$),
so using this lower bound in integration and switching to radial
coordinates, we have
\begin{align*}
  K_1^{(\delta)}
  \; \ge \; \iint_{B_\delta} \frac{C_1}{\|\theta\|^2} \; \ud^d \theta
  \; = \; C_1 \, A_d \int_{0}^{\delta} \rho^{d-3} \; \ud \rho
  \; = \;  + \infty .
\end{align*}
\end{proof}

\begin{lemma}
  \label{lem:SRW_low_frequency_divergence}
  \uses{def:frequency_decomposition}
  For the simple random walk $\RW = \big(\RW(t)\big)_{t \in \bN}$ on $\bZ^d$
  with $d \le 2$,
  we have $\lim_{r \nearrow 1} \, K_r^{(\delta)} \, = \, +\infty$
  for any $\delta > 0$ small.
\end{lemma}
\begin{proof}
  \uses{lem:SRW_low_frequency_approximation}
  By the increasing limit part of
  Lemma~\ref{lem:SRW_low_frequency_approximation} and
  monotone convergence theorem, we have
  \begin{align*}
  K_r^{(\delta)} \uparrow K_1^{(\delta)}
    = \iint_{B_\delta} \re \big( \Freg{1} (\theta) \big) \; \ud^d \theta .
  \end{align*}
  By the second part of Lemma~\ref{lem:SRW_low_frequency_approximation},
  $\re \big( \Freg{1}(\theta) \big) \le C_2 \| \theta \|^{-2}$ (for some positive constant $C_2$),
  so using this upper bound in integration and switching to radial
  coordinates, we have
  \begin{align*}
    K_1^{(\delta)}
    \; \le \; \iint_{B_\delta} \frac{C_2}{\|\theta\|^2} \; \ud^d \theta
    \; = \; C_2 \, A_d \int_{0}^{\delta} \rho^{d-3} \; \ud \rho
    \; < \;  + \infty .
  \end{align*}
\end{proof}
