\chapter{Treatment of the integral in the Fourier inversion}

In this part, we analyze the integral
\begin{align}
I_r = \iint_{\Fbox} \Freg{r} (\theta) \; \ud^d \theta ,
\end{align}
where $\Freg{r} \colon \bR^d \to \bC$ is the Fourier transform of the
regularized Green's function of a random walk $\RW$ on $\bZ^d$. By
Corollary~\ref{cor:recurrence_iff_finite_limit_integral}, the finiteness
of this integral in the limit $r \nearrow 1$ characterizes expectation
transience of $\RW$.

\section{Decomposition of the integral}

\begin{lemma}
  \label{lem:integral_decomposition}
  \uses{def:Green_Fourier_transf, lem:Green_Fourier_inverse_real}
  For any $0 < \delta \le \pi$ we can write
  \begin{align*}
  I_r = J_r^{(\delta)} + K_r^{(\delta)},
  \end{align*}
  where the two parts are
  \begin{align}
  J_r^{(\delta)} = \; & \iint_{\Fbox \setminus B_\delta} \Freg{r} (\theta) \; \ud^d \theta \\
  K_r^{(\delta)} = \; & \iint_{B_\delta} \Freg{r} (\theta) \; \ud^d \theta ,
  \end{align}
  where $B_\delta := \set{ \theta \in \bR^d \, \big| \, \|\theta\| < \delta}$ is the
  ball of radius $\delta$ centered at $\vec{0} \in \bR^d$.
\end{lemma}
\begin{proof}
Obvious, since $\Fbox = (\Fbox \setminus B_\delta) \cup B_\delta$
is a disjoint union.
\end{proof}


\section{Dominated convergence away from the origin}

TODO: Define non-degenerate step distribution (essentially $\sum_{u\in \bZ^d} p(u) e^{\ii u \cdot \theta} \ne 1$ for $\theta \ne 0$ modulo periodicity).

\begin{lemma}
  \label{lem:integral_away}
  \uses{lem:integral_decomposition}
  If $\RW$ is a time-homogeneous Markovian random walk with suitable
  non-degeneracy conditions on its step distribution (to be written down more precisely),
  then for any $0 < \delta \le \pi$ the limit
  \begin{align*}
  \lim_{r \nearrow 1} J_r^{(\delta)}
  \end{align*}
  exists and is finite (limit in $\bR$).
\end{lemma}
\begin{proof}
Under the nondegeneracy conditions, on the compact set $\Fbox \setminus B_\delta$,
the continuous integrand $\theta \mapsto \Freg{r}(\theta)$
is bounded (and therefore dominated by a constant function) and
it has the pointwise limit $\lim_{r \nearrow 1} \Freg{r}(\theta) = \Freg{1}(\theta)$.
It therefore follows from the dominated convergence theorem
that $\lim_{r \nearrow 1} J_r^{(\delta)} = J_1^{(\delta)} \in \bR$.
\end{proof}

\section{Monotone convergence near the origin}

TODO: Think about the best conditions for step distribution under which monotone convergence can be applied (real-valuedness requires symmetricity of the step-distribution?!?).

\begin{lemma}
  \label{lem:integral_near}
  \uses{lem:integral_decomposition}
  If $\RW$ is a time-homogeneous Markovian random walk with suitable
  symmetricity and integrability conditions on its step distribution (to be written down more precisely),
  then there exists a $\delta_0 > 0$ such that for any $0 < \delta \le \delta_0$, the limit
  \begin{align*}
  \lim_{r \nearrow 1} K_r^{(\delta)}
  \end{align*}
  is increasing and exists in $[0,+\infty]$.
\end{lemma}
\begin{proof}
\ldots
\end{proof}


\section{Characterizing finiteness of the integral}
