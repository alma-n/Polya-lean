\chapter{Recurrence and transience}

Let $\RW = \big(\RW(t)\big)_{t \in \bN}$ be a random walk on a graph with
vertex set $\Gr$, starting from $x_0 \in \Gr$.

\section{Basic definition}

\begin{definition}%[Recurrence]
  \label{def:recurrence}
  %\uses{def:walk}
  %\lean{RW}
  %\leanok
  %A random walk $\RW = \big(\RW(t)\big)_{t \in \bN}$ on a graph with
  %vertex set $\Gr$, starting from $x_0 \in \Gr$,
  The random walk $\RW = \big(\RW(t)\big)_{t \in \bN}$
  is said to be \textbf{recurrent} if
  \begin{align*}
  \PR\Big[ X(t) = x_0 \text{ for infinitely many } t \in \bN \Big] \, = \, 1
  \end{align*}
  and \textbf{transient} if
  \begin{align*}
  \PR\Big[ X(t) = x_0 \text{ for infinitely many } t \in \bN \Big] \, = \, 0 .
  \end{align*}

\end{definition}

\section{Equivalent conditions}

Usually random walks are taken to be Markov processes (Markovian random walks).
Then one can use alternative formulations of recurrence and transience.

\begin{definition}%[Expectation recurrence]
  \label{def:expectation_recurrence}
  %\uses{def:walk}
  %\lean{RW}
  %\leanok
  %Let $\RW = \big(\RW(t)\big)_{t \in \bN}$ be a random walk on a graph with
  %vertex set $\Gr$, starting from $x_0 \in \Gr$.
  Denote by
  $L = \# \set{t \in \bN \, \Big| \, X(t) = x_0}$ the number of times
  the random walk $\RW = \big(\RW(t)\big)_{t \in \bN}$ is at its starting point.
  The random walk $X$ is said to be \textbf{expectation recurrent} if
  $\EX [ L ] \, = \, +\infty$
  and \textbf{expectation transient} if
  $\EX [ L ] \; < \; +\infty$.
\end{definition}

\begin{definition}%[First return recurrence]
  \label{def:Markovian_recurrence}
  %\uses{def:walk}
  %\lean{RW}
  %\leanok
  %A random walk $\RW = \big(\RW(t)\big)_{t \in \bN}$ on a graph with
  %vertex set $\Gr$, starting from $x_0 \in \Gr$,
  The random walk $\RW = \big(\RW(t)\big)_{t \in \bN}$
  is said to be \textbf{first return recurrent} if
  $\PR\big[ X(t) = x_0 \text{ for some } t > 0 \big] \, = \, 1$
  and \textbf{first return transient} if
  $\PR\big[ X(t) = x_0 \text{ for some } t > 0 \big] \; < \; 1$.
\end{definition}

\begin{lemma}
  \label{lem:recurrent_iff_expectation_recurrent}
  \uses{def:recurrence, def:expectation_recurrence, def:iid_random_walk}
  A Markovian random walk $\RW = \big(\RW(t)\big)_{t \in \bN}$ is recurrent
  %(in the sense of Definition~\ref{def:recurrence})
  if and only if it is
  expectation recurrent.
  %(in the sense of Definition~\ref{def:expectation_recurrence}).
\end{lemma}
\begin{proof}
\ldots
\end{proof}

\begin{lemma}
  \label{lem:recurrent_iff_return_recurrent}
  \uses{def:recurrence, def:Markovian_recurrence, def:iid_random_walk}
  A Markovian random walk $\RW = \big(\RW(t)\big)_{t \in \bN}$ is recurrent
  %(in the sense of Definition~\ref{def:recurrence})
  if and only if it is
  first return recurrent.
  %(in the sense of Definition~\ref{def:Markovian_recurrence}).
\end{lemma}
\begin{proof}
\ldots
\end{proof}

Simple random walks are Markovian, so the previous two lemmas imply that
Definitions~\ref{def:recurrence},
\ref{def:expectation_recurrence} and \ref{def:Markovian_recurrence} are logically
equivalent for a simple random walk.
This project first establishes expectation recurrence
in Theorem~\ref{thm:Polya_alt}; the other
formulations are obtained as consequences.
